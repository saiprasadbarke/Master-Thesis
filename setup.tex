
%------------------------------------------------------------------------------
%       package includes
%------------------------------------------------------------------------------
    % font encoding is set up for pdflatex, for other environments see
    % http://tex.stackexchange.com/questions/44694/fontenc-vs-inputenc
    \usepackage[T1]{fontenc}  % 8-bit fonts, improves handling of hyphenations
    \usepackage[utf8x]{inputenc}
    % provides `old' commands for table of contents. Eases the ability to switch
    % between book and scrbook
    \usepackage{scrhack}
    \usepackage[section]{placeins}

    % ------------------- layout, default -------------------
    % adjust the style of float's captions, separated from text to improve readabilty
    \usepackage[labelfont=bf, labelsep=colon, format=hang, textfont=singlespacing]{caption}
    \usepackage{chngcntr}  % continuous numbering of figures/tables over chapters
    \counterwithout{equation}{chapter}
    \counterwithout{figure}{chapter}
    \counterwithout{table}{chapter}

    % Uncomment the following line if you switch from scrbook to book
    % and comment the setkomafont line
    %\usepackage{titlesec}  % remove "Chapter" from the chapter title
    %\titleformat{\chapter}[hang]{\bfseries\huge}{\thechapter}{2pc}{\huge}
    \setkomafont{chapter}{\normalfont\bfseries\huge}

    \usepackage{setspace}  % Line spacing
    \onehalfspacing
    % \doublespacing  % uncomment for double spacing, e.g. for annotations in correction

    % ------------------- functional, default-------------------
    \usepackage{array}  % custom format per column in table - needed on the title page
    \usepackage{graphicx}  % include graphics
    \usepackage{subfig}  % divide figure, e.g. 1(a), 1(b)...
    \usepackage{amsmath}  % |
    \usepackage{amsthm}   % | math, bmatrix etc
    \usepackage{amsfonts} % |
    \usepackage{calc}  % calculate within LaTeX
    \usepackage[unicode=true,bookmarks=true,bookmarksnumbered=true,
                bookmarksopen=true,bookmarksopenlevel=1,breaklinks=false,
                pdfborder={0 0 0},backref=false,colorlinks=false]{hyperref}

    \usepackage{fancyhdr}
    \usepackage{longtable}
    %==========================================
    % You might not need the following packages, I only included them as they
    % are needed for the example floats
    % ------------------- functional, custom -------------------
    \usepackage{algorithm,algpseudocode}
    \usepackage{bm}  % bold greek variables (boldmath)
    \usepackage{tikz}
    \usetikzlibrary{positioning}  % use: above left of, etc

    % Improves general appearance of the text
    \usepackage[protrusion=true,expansion=true, kerning]{microtype}
    % Include other packages here, before hyperref.
    
    \usepackage{adjustbox}
    \usepackage{graphicx}
    \usepackage{amsmath}
    \usepackage{amssymb}
    \usepackage{amsfonts}
    \usepackage{booktabs}
    \usepackage{times}
    \usepackage{epsfig}
    % \usepackage{adjcalc}
    % \usepackage{adjustbox}
    \usepackage{pifont}
    \usepackage{color, colortbl}
    \usepackage{siunitx}
    % \usepackage{subfig}
    % \usepackage{bm}
    \usepackage[table, dvipsnames]{xcolor} % for coloring cells
    % \usepackage{blindtext}
    % \usepackage{lipsum}
    % \usepackage{mathtools} 
    \usepackage{color, colortbl}
    \usepackage{tabularx,ltablex}
    \usepackage{arydshln}
    \usepackage{ragged2e}
% It is strongly recommended to use hyperref, especially for the review version.
% hyperref with option pagebackref eases the reviewers' job.
% Please disable hyperref *only* if you encounter grave issues, e.g. with the
% file validation for the camera-ready version.
%
% If you comment hyperref and then uncomment it, you should delete
% ReviewTempalte.aux before re-running LaTeX.
% (Or just hit 'q' on the first LaTeX run, let it finish, and you
%  should be clear).


% Support for easy cross-referencing
\usepackage[capitalize]{cleveref}

%------------------------------------------------------------------------------
%       (re)new commands / settings
%------------------------------------------------------------------------------
    % ----------------- referencing ----------------
    \newcommand{\secref}[1]{Section~\ref{#1}}
    \newcommand{\chapref}[1]{Chapter~\ref{#1}}
    \renewcommand{\eqref}[1]{Equation~(\ref{#1})}
    \newcommand{\figref}[1]{Figure~\ref{#1}}
    \newcommand{\tabref}[1]{Table~\ref{#1}}

    % ------------------- colors -------------------
    \definecolor{darkgreen}{rgb}{0.0, 0.5, 0.0}
    % Colors of the Albert Ludwigs University as in
    % https://www.zuv.uni-freiburg.de/service/cd/cd-manual/farbwelt
    \definecolor{UniBlue}{RGB}{0, 74, 153}
    \definecolor{UniRed}{RGB}{193, 0, 42}
    \definecolor{UniGrey}{RGB}{154, 155, 156}


    % ------------------- layout -------------------
    % prevents floating objects from being placed ahead of their section
    \let\mySection\section\renewcommand{\section}{\suppressfloats[t]\mySection}
    \let\mySubSection\subsection\renewcommand{\subsection}{\suppressfloats[t]\mySubSection}


    % ------------------- marker commands -------------------
    % ToDo command
    \newcommand{\todo}[1]{\textbf{\textcolor{red}{(TODO: #1)}}}
    \newcommand{\extend}[1]{\textbf{\textcolor{darkgreen}{(EXTEND: #1)}}}
    % Lighter color to note down quick drafts
    \newcommand{\draft}[1]{\textbf{\textcolor{NavyBlue}{(DRAFT: #1)}}}


    % ------------------- math formatting commands -------------------
    % define vectors to be bold instead of using an arrow
    \renewcommand{\vec}[1]{\mathbf{#1}}
    \newcommand{\mat}[1]{\mathbf{#1}}
    % tag equation with name
    \newcommand{\eqname}[1]{\tag*{#1}}


    % ------------------- pdf settings -------------------
    % ADAPT THIS
    \hypersetup{pdftitle={Analysing Multi-view Depth Estimation in a common framework},
                pdfauthor={Saiprasad Barke},
                pdfsubject={Master thesis at the University of Freiburg},
                pdfkeywords={deep learning, multi-view stereo,  master thesis},
                pdfpagelayout=OneColumn, pdfnewwindow=true, pdfstartview=XYZ, plainpages=false}


    %==========================================
    % You might not need the following commands, I only included them as they
    % are needed for the example floats

    % ------------------- Tikz styles -------------------
    \tikzset{>=latex}  % arrow style


    % ------------------- algorithm ---------------------
    % Command to align comments in algorithm
    \newcommand{\alignedComment}[1]{\Comment{\parbox[t]{.35\linewidth}{#1}}}
    % define a foreach command in algorithms
    \algnewcommand\algorithmicforeach{\textbf{foreach}}
    \algdef{S}[FOR]{ForEach}[1]{\algorithmicforeach\ #1\ \algorithmicdo}

    
    \crefname{section}{Sec.}{Secs.}
    \Crefname{section}{Section}{Sections}
    \Crefname{table}{Table}{Tables}
    \crefname{table}{Tab.}{Tabs.}
    
    % colors:
    \definecolor{mylightgreen}{RGB}{51,153,102}
    \definecolor{mydarkgreen}{RGB}{17,121,74}
    
    \definecolor{mygreenblue}{RGB}{47,111,112}
    \definecolor{mylightblue}{RGB}{172,207,208} 
    
    \definecolor{myyellow}{RGB}{215,179,18} 
    \definecolor{myorange}{RGB}{215,136,18}
    
    \definecolor{mylightred}{RGB}{192,72,72} 
    \definecolor{mydarkred}{RGB}{158,28,28} 
    
    \definecolor{mylightgray}{RGB}{238,238,238} 
    \definecolor{mymedgray}{RGB}{202,204,206} 
    \definecolor{mydarkgray}{RGB}{42,44,46}
    
    \definecolor{mymaxturbo}{RGB}{161,17,1}
    \definecolor{myminturbo}{RGB}{57,42,119}
    
    \colorlet{bgcolor}{mylightgray}
    \colorlet{poscolor}{mydarkgreen}
    \colorlet{negcolor}{mydarkred}
    \colorlet{outputscalingcolor}{blue}
    \definecolor{draftcolor}{RGB}{0,73,95}

    % comment commands:
    \newcommand{\tb}[1]{\textcolor{mydarkgreen}{\footnotesize Thomas: #1}} % Thomas comments
    \newcommand{\tcb}[1]{\textcolor{mydarkgreen}{\footnotesize Thomas: #1}} % Thomas comments
    \newcommand{\tcp}[1]{\textcolor{mydarkred}{\footnotesize Philipp: #1}} % Philipp comments
    \newcommand{\cz}[1]{\textcolor{mygreenblue}{\footnotesize Christian: #1}} % Christian comments
    \newcommand{\tco}[1]{\textcolor{myorange}{\footnotesize Ozgun: #1}} % Ozgun comments
    \newcommand{\ar}[1]{\textcolor{mygreenblue}{\footnotesize Artemij: #1}} % Christian comments
    \newcommand{\jan}[1]{\textcolor{mylightgreen}{\footnotesize Jan: #1}} % Jan comments
    
    % \newcommand{\tb}[1]{\textcolor{mydarkgreen}{}} % Thomas comments
    % \newcommand{\tcb}[1]{\textcolor{mydarkgreen}{}} % Thomas comments
    % \newcommand{\tcp}[1]{\textcolor{mydarkred}{}} % Philipp
    % \newcommand{\cz}[1]{\textcolor{mygreenblue}{}} % Christian comments
    % \newcommand{\tco}[1]{\textcolor{myorange}{}} % Ozgun comments
    % \newcommand{\ar}[1]{\textcolor{mygreenblue}{}} % Christian comments
    % \newcommand{\jan}[1]{\textcolor{mylightgreen}{}} % Jan comments
    
    \newcommand\blfootnote[1]{%
      \begingroup
      \renewcommand\thefootnote{}\footnote{#1}%
      \addtocounter{footnote}{-1}%
      \endgroup
    }
    
    \newcommand{\tdraft}[1]{\textcolor{draftcolor}{Draft: #1}}
    \newcommand{\ttodo}[1]{\textcolor{mydarkred}{Todo: #1}}
    
    % custom commands:
    
    \newcommand{\fakeparagraph}[1]{\noindent\textbf{#1}{ }}
    
    \newcommand{\abs}[1]{\left\lvert #1 \right\rvert}
    \newcommand{\norm}[2]{\left\lVert #1 \right\rVert_{#2}}
    \newcommand{\loss}{\mathcal{L}}
    \newcommand{\vect}[1]{\mathbf{#1}}
    % \newcommand{\my}{\textcolor{poscolor}{\ding{51}}}
    % \newcommand{\mn}{\textcolor{negcolor}{\ding{51}}}
    \newcommand{\my}{\ding{51}}
    \newcommand{\mn}{\ding{55}}
    \newcommand{\bb}[1]{\textbf{#1}}
    \def\rot#1{\rotatebox{90}{#1}}
    \DeclareMathOperator*{\argmax}{arg\,max}
    \DeclareMathOperator*{\argmin}{arg\,min}
    \DeclareMathOperator*{\minimize}{minimize}
    \DeclareMathOperator*{\maximize}{minimize}
    
    \DeclareMathOperator{\sign}{sign}
    \DeclareMathOperator{\rank}{rank}
    \DeclareMathOperator{\trace}{tr}
    \DeclareMathOperator{\diag}{diag}
    \DeclareMathOperator{\diam}{diam}
    
    % placeholders:
    \newcommand{\rmvd}{RobustMVD}
    \newcommand{\mvsn}{MVSNet}
    \newcommand{\vmn}{Vis-MVSNet}
    \hyphenation{DistNet}
    
    % math placeholders naming scheme:
        % -set of sets: mathcal
        % -set/tuple: uppercase
        % -matrix: uppercase, bold
        % -vector: lowercase, bold
        % scalar: lowercase
        % function (e.g. neural network): lowercase (or uppercase?)
    
    \newcommand{\other}{source}
    \newcommand{\Other}{Source}
    \newcommand{\Otherview}{\Other{} view}
    \newcommand{\otherview}{\other{} view}
    \newcommand{\otherviews}{\otherview{}s}
    \newcommand{\key}{key}
    \newcommand{\Key}{Key}
    \newcommand{\keyview}{keyview}
    \newcommand{\keyviews}{keyviews}
    
    \newcommand{\sethree}{SE(3)}
    \newcommand{\sothree}{SO(3)}
    \newcommand{\realnumbers}{\mathbb{R}}
    \newcommand{\projectivespace}{\mathbb{P}}
    
    \newcommand{\fromto}[2]{{}^{#1}_{#2}}
    \newcommand{\coordsin}[2]{{}^{#1}{#2}}
    \newcommand{\homog}[1]{\tilde{#1}}
    \newcommand{\scaledshifted}[1]{\hat{#1}}
    \newcommand{\scaledshiftedin}[2]{\coordsin{#1}{\scaledshifted{#2}}}
    
    \newcommand{\testsets}{\mathcal{D}}
    \newcommand{\testset}{D}
    \newcommand{\sample}{S}
    \newcommand{\sampleinput}{V}
    \newcommand{\labels}{L}
    \newcommand{\view}{V}
    
    \newcommand{\inputwidth}{W}
    \newcommand{\inputheight}{H}
    \newcommand{\featurewidth}{w}
    \newcommand{\featureheight}{h}
    \newcommand{\featurechannels}{c}
    \newcommand{\correlationsteps}{n}
    
    \newcommand{\image}{\mathbf{I}}
    \newcommand{\featuremap}{\mathbf{F}}
    \newcommand{\ctxfeaturemap}{\mathbf{\hat{F}}}
    \newcommand{\feature}{\mathbf{f}}
    \newcommand{\intrinsic}{\mathbf{K}}
    \newcommand{\costvolume}{\textbf{C}}
    % \newcommand{\encoder}{f}
    % \newcommand{\fusion}{g}
    % \newcommand{\decoder}{h}
    % \newcommand{\ctx}{k}
    \newcommand{\encoder}{f_{\theta}}
    \newcommand{\fusion}{g_{\rho}}
    \newcommand{\ctx}{h_{\sigma}}
    \newcommand{\decoder}{k_{\phi}}
    \newcommand{\uncertaintydecoder}{l_{\psi}}
    \newcommand{\simfct}{sim}
    
    \newcommand{\transform}{\mathbf{T}}
    \newcommand{\translation}{\mathbf{t}}
    \newcommand{\rotation}{\mathbf{R}}
    \newcommand{\transformfromto}[2]{\fromto{#1}{#2}\transform}
    \newcommand{\translationfromto}[2]{\fromto{#1}{#2}\translation}
    \newcommand{\rotationfromto}[2]{\fromto{#1}{#2}\rotation}
    
    % uncertainty:
    \newcommand{\uncertaintymap}{\mathbf{U}}  % 2d matrix of depth values
    
    % depth:
    \newcommand{\depthval}{z}
    \newcommand{\depths}{\mathbf{\depthval}}  % vector of depth values
    \newcommand{\depthmap}{\mathbf{Z}}  % 2d matrix of depth values
    
    \newcommand{\preddepthval}{\depthval}
    \newcommand{\preddepths}{\depths}
    \newcommand{\preddepthmap}{\depthmap}
    \newcommand{\preddepthsscaled}{\hat{\preddepths}}  % or\mathbf{\hat{z}}
    \newcommand{\preddepthmapscaled}{\hat{\preddepthmap}}  % or\mathbf{\hat{z}}
    
    \newcommand{\gtdepthval}{\depthval^\ast}
    \newcommand{\gtdepths}{\depths^\ast}  % or \mathbf{\gtdepthval} 
    \newcommand{\gtdepthmap}{\depthmap^\ast}
    
    %invdepth:
    \newcommand{\invdepthval}{d}
    \newcommand{\invdepths}{\mathbf{\invdepthval}}  % vector of invdepth values
    \newcommand{\invdepthmap}{\mathbf{D}}  % 2d matrix of invdepth values
    \newcommand{\invdepthvalin}[1]{\coordsin{#1}{\invdepthval}}
    
    \newcommand{\predinvdepthval}{\invdepthval}
    \newcommand{\predinvdepths}{\invdepths}
    \newcommand{\predinvdepthmap}{\invdepthmap}
    
    \newcommand{\gtinvdepthval}{\invdepthval^\ast}
    \newcommand{\gtinvdepths}{\invdepths^\ast}  % or \mathbf{\gtinvdepthval}
    \newcommand{\gtinvdepthmap}{\invdepthmap^\ast}
    
    % disparity:
    \newcommand{\dispmap}{\bm{\Delta}}
    \newcommand{\dispmapin}[1]{\coordsin{#1}\dispmap}
    
    \newcommand{\disp}{\bm{\delta}}
    \newcommand{\homogdisp}{\homog{\bm{\Delta}}}
    \newcommand{\dispin}[1]{\coordsin{#1}\disp}
    \newcommand{\homogdispin}[1]{\coordsin{#1}\homogdisp}
    
    \newcommand{\dispval}{\delta}
    \newcommand{\homogdispval}{\homog{\mathbf{m}}}
    \newcommand{\dispvalin}[1]{\coordsin{#1}\dispval}
    \newcommand{\homogdispvalin}[1]{\coordsin{#1}\homogdispval}
    \newcommand{\gtdispval}{\dispval^\ast}
    \newcommand{\gtdispvalin}[1]{\coordsin{#1}\gtdispval}
    
    \newcommand{\dispvalx}{\dispval_{\pointtwodx}}
    \newcommand{\dispgradx}{\dispvalx'}
    \newcommand{\homogdispvalx}{\homog{m_{\pointtwodx}}}
    \newcommand{\dispvalxin}[1]{\coordsin{#1}\dispvalx}
    \newcommand{\dispgradxin}[1]{\coordsin{#1}\dispgradx}
    \newcommand{\homogdispvalxin}[1]{\coordsin{#1}\homogdispvalx}
    
    \newcommand{\dispvaly}{\dispval_{\pointtwody}}
    \newcommand{\dispgrady}{\dispvaly'}
    \newcommand{\homogdispvaly}{\homog{m_{\pointtwody}}}
    \newcommand{\dispvalyin}[1]{\coordsin{#1}\dispvaly}
    \newcommand{\dispgradyin}[1]{\coordsin{#1}\dispgrady}
    \newcommand{\homogdispvalyin}[1]{\coordsin{#1}\homogdispvaly}
    
    \newcommand{\dispvalz}{\dispval_{\pointtwodz}}
    \newcommand{\homogdispvalz}{\homog{m_{\pointtwodz}}}
    \newcommand{\dispvalzin}[1]{\coordsin{#1}\dispvalz}
    \newcommand{\homogdispvalzin}[1]{\coordsin{#1}\homogdispvalz}
    
    % points in 2d and 3d:
    
    \newcommand{\pointthreed}{\mathbf{X}}
    \newcommand{\pointthreedin}[1]{\coordsin{#1}{\pointthreed}}
    \newcommand{\homogpointthreed}{\homog{\pointthreed}}
    \newcommand{\homogpointthreedin}[1]{\coordsin{#1}{\homogpointthreed}}
    
    \newcommand{\pointthreedx}{x}
    \newcommand{\pointthreedxin}[1]{\coordsin{#1}{\pointthreedx}}
    \newcommand{\pointthreedy}{y}
    \newcommand{\pointthreedyin}[1]{\coordsin{#1}{\pointthreedy}}
    \newcommand{\pointthreedz}{z}
    \newcommand{\pointthreedzin}[1]{\coordsin{#1}{\pointthreedz}}
    
    \newcommand{\pointtwod}{\mathbf{x}}
    \newcommand{\pointtwodin}[1]{\coordsin{#1}{\pointtwod}}
    \newcommand{\homogpointtwod}{\homog{\pointtwod}}
    \newcommand{\homogpointtwodin}[1]{\coordsin{#1}{\homogpointtwod}}
    
    \newcommand{\pointtwodx}{u}
    \newcommand{\pointtwodxin}[1]{\coordsin{#1}{\pointtwodx}}
    \newcommand{\homogpointtwodx}{\homog{\pointtwodx}}
    \newcommand{\homogpointtwodxin}[1]{\coordsin{#1}{\homogpointtwodx}}
    \newcommand{\pointtwody}{v}
    \newcommand{\pointtwodyin}[1]{\coordsin{#1}{\pointtwody}}
    \newcommand{\homogpointtwody}{\homog{\pointtwody}}
    \newcommand{\homogpointtwodyin}[1]{\coordsin{#1}{\homogpointtwody}}
    \newcommand{\pointtwodz}{k}
    \newcommand{\homogpointtwodz}{\homog{\pointtwodz}}
    \newcommand{\homogpointtwodzin}[1]{\coordsin{#1}{\homogpointtwodz}}
    
    % %%%%%%%%%%%%%%%%%%%%%%%%%%%%%%%%%%%%%%%%%%%%%%%%%%%%%%%%%%%%%%%%%%%%%%%%%%%%%%%%%%%%%%%%
    
    \newcommand{\pointalpha}{\mathbf{\pmb{\alpha}}}
    \newcommand{\pointalphain}[1]{\coordsin{#1}{\pointalpha}}
    
    \newcommand{\pointalphax}{\alpha_\pointtwodx}
    \newcommand{\pointalphaxin}[1]{\coordsin{#1}{\pointalphax}}
    \newcommand{\pointalphay}{\alpha_\pointtwody}
    \newcommand{\pointalphayin}[1]{\coordsin{#1}{\pointalphay}}
    
    % %%%%%%%%%%%%%%%%%%%%%%%%%%%%%%%%%%%%%%%%%%%%%%%%%%%%%%%%%%%%%%%%%%%%%%%%%%%%%%%%%%%%%%%%
    
    \newcommand{\pointtwodinf}{\pointtwod_{\infty}}
    \newcommand{\pointtwodinfin}[1]{\coordsin{#1}{\pointtwodinf}}
    \newcommand{\homogpointtwodinf}{\homog{\pointtwod}_{\infty}}
    \newcommand{\homogpointtwodinfin}[1]{\coordsin{#1}{\homogpointtwodinf}}
    
    \newcommand{\pointtwodinfx}{\pointtwodx_{\infty}}
    \newcommand{\pointtwodinfxin}[1]{\coordsin{#1}{\pointtwodinfx}}
    \newcommand{\homogpointtwodinfx}{\homog{\pointtwodx}_{\infty}}
    \newcommand{\homogpointtwodinfxin}[1]{\coordsin{#1}{\homogpointtwodinfx}}
    
    \newcommand{\pointtwodinfy}{\pointtwody_{\infty}}
    \newcommand{\pointtwodinfyin}[1]{\coordsin{#1}{\pointtwodinfy}}
    \newcommand{\homogpointtwodinfy}{\homog{\pointtwody}_{\infty}}
    \newcommand{\homogpointtwodinfyin}[1]{\coordsin{#1}{\homogpointtwodinfy}}
    
    \newcommand{\pointtwodinfz}{\pointtwodz_{\infty}}
    \newcommand{\pointtwodinfzin}[1]{\coordsin{#1}{\pointtwodinfz}}
    \newcommand{\homogpointtwodinfz}{\homog{\pointtwodz}_{\infty}}
    \newcommand{\homogpointtwodinfzin}[1]{\coordsin{#1}{\homogpointtwodinfz}}
    
    % shifts and scales:
    \newcommand{\scalingfactor}{\alpha}
    \newcommand{\shift}{b}
    \newcommand{\scalingfactorin}[1]{\coordsin{#1}{\scalingfactor}}
    \newcommand{\shiftin}[1]{\coordsin{#1}{\shift}}
    \newcommand{\scalemap}{\mathbf{A}}
    \newcommand{\shiftmap}{\mathbf{B}}
    
    % metrics:
    \newcommand{\ausename}{Area Under Sparsification Error Curve}
    \newcommand{\ause}{\textrm{AUSE}}
    \newcommand{\absrelname}{Absolute Relative Error}
    % \newcommand{\absrel}{AbsRel}
    \newcommand{\absrel}{\textrm{rel}}
    \newcommand{\sqrel}{SqRel}
    \newcommand{\sqrelsign}{SqRel}
    \newcommand{\rmse}{RMSE}
    \newcommand{\rmsesign}{RMSE}
    \newcommand{\rmselog}{RMSElog}
    \newcommand{\rmselogsign}{RMSElog}
    \newcommand{\threshIname}{Inlier Ratio}% with a threshold of $1.25$}
    % \newcommand{\threshI}{\delta_{<1.25}}
    \newcommand{\threshI}{\tau}
    \newcommand{\threshII}{$\delta<1.25^2$}
    \newcommand{\threshIIsign}{$\delta<{1.25^2}$}
    \newcommand{\threshIII}{$\delta<1.25^3$}
    \newcommand{\threshIIIsign}{$\delta<{1.25^3}$}
    
    % datasets:
    \newcommand{\kitti}{KITTI}
    \newcommand{\kittishort}{KITTI}
    \newcommand{\megadepth}{MegaDepth}
    \newcommand{\megadepthshort}{MD}
    \newcommand{\redweb}{ReDWeb}
    \newcommand{\redwebshort}{RW}
    \newcommand{\movies}{3D Movies}
    \newcommand{\moviesshort}{MV}
    \newcommand{\sintel}{Sintel}
    \newcommand{\sintelshort}{Sintel}
    \newcommand{\diw}{Depth in the Wild}
    \newcommand{\diwshort}{DIW}
    \newcommand{\diml}{DIML Indoor}
    \newcommand{\dimlshort}{DL}
    \newcommand{\wsvd}{Web Stereo Video Dataset}
    \newcommand{\wsvdshort}{WSVD}
    \newcommand{\ethd}{ETH3D}
    \newcommand{\ethdshort}{ETH3D}
    \newcommand{\nyu}{NYUDv2}
    \newcommand{\nyushort}{NYU}
    \newcommand{\tumrgbd}{TUM-RGBD}
    \newcommand{\tumrgbdshort}{TUM}
    \newcommand{\scannet}{ScanNet}
    \newcommand{\scannetshort}{ScanNet}
    \newcommand{\flyingthings}{FlyingThings3D}
    \newcommand{\flyingthingsshort}{FT3D}
    \newcommand{\staticthings}{StaticThings3D}
    \newcommand{\staticthingsshort}{ST3D}
    \newcommand{\realthings}{RealThings}
    \newcommand{\realthingsshort}{RT}
    \newcommand{\sun}{SUN3D}
    \newcommand{\sunshort}{SUN}
    \newcommand{\dtu}{DTU}
    \newcommand{\dtushort}{DTU}
    \newcommand{\tanksandtemples}{Tanks and Temples}
    \newcommand{\tanksandtemplesshort}{T\&T}
    \newcommand{\blendedmvs}{BlendedMVS}
    \newcommand{\blendedmvsshort}{BMVS}
    
    \newcommand{\benchmarkname}{Robust MVD Benchmark}
    \newcommand{\baselinename}{Robust MVD Baseline}
    \newcommand{\framework}{Robust Multi-View Depth}
    \newcommand{\mv}{multi-view}
    \newcommand{\mvs}{Multi-View Stereo}
    \newcommand{\dispn}{DispNet}
    % results table:
    % \newcommand{\trainedsimto}[1]{\textcolor{negcolor}{#1}}
    \newcommand{\trainedsimto}[1]{(#1)}
    \newcommand{\bestresult}[1]{\textbf{#1}}
    \newcommand{\usesoutputscaling}[1]{\textcolor{outputscalingcolor}{#1}}
    \newcommand{\worsethanmedian}[1]{\textcolor{negcolor}{#1}}
    
    
    \newcommand{\overbar}[1]{\mkern 1.5mu\overline{\mkern-1.5mu#1\mkern-1.5mu}\mkern 1.5mu}
    \newcommand{\emphbf}[1]{\emph{\textbf{#1}\xspace}}
    \newcommand{\mypara}[1]{\smallskip\noindent\emphbf{#1.}\xspace}
    \newcommand\inv[1]{#1\raisebox{1.15ex}{$\scriptscriptstyle-\!1$}}

    \newcommand{\etal}{\textit{et al.}}
    \newcommand{\bms}{\textit{ BlendedMVS-MVSNet variant}}
    \newcommand{\brs}{\textit{ BlendedMVS-RMVD variant}}
    \newcommand{\gwc}{Groupwise correlation}

    \newcolumntype{M}[1]{>{\centering\arraybackslash}m{#1}}