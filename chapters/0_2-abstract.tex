\chapter*{Abstract}
\paragraph{}Understanding the key factors influencing the performance of depth estimation models is crucial in advancing computer vision applications such as 3D reconstruction, autonomous navigation, and augmented reality. This thesis explores these influencing factors within the context of the \framework {(\rmvd)}\cite{schroeppel2022benchmark} framework. We implement several models under the {\rmvd} umbrella, each uniquely different in architectural elements, to decipher their individual and collective impacts on depth estimation performance. We also examine the contributions of different data augmentation techniques and investigate the implications of the choice of training datasets. Further, we scrutinize the effects of manipulating data properties, like the number of source views, source selection approach, and diverse training hyperparameters, on the model's predictive performance and uncertainty estimation, offering insights into the principles that underlie the performance of {\mv} depth estimation models to empower future model design and refinement of depth estimation methodologies. \par
We observe that the depth estimation performance within the {\rmvd} framework is substantially influenced by the selection of the correlation layer and the feature extractor's capability to produce representative features. Additionally, the choice of input data and the way it's processed plays an indispensable role in model robustness. We also underscore the importance of normalizing features before correlation, a simple yet effective enhancement. The research also sheds light on potential avenues for future exploration, emphasizing the myriad of methodologies that, when aptly combined, can significantly boost the performance of {\mvs} models.
This report outlines the experimental setup, presents the results of the ablation study, and discusses potential directions for future research in the domain of depth estimation in a {\mv} setting. 